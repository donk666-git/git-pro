% ===================================================================
% LaTeX 作业主文件 (最终优化版)
% ===================================================================

\documentclass[12pt, a4paper, twoside]{ctexart}

\AtBeginDocument{%
  \abovedisplayskip=6pt plus 2pt minus 2pt
  \belowdisplayskip=6pt plus 2pt minus 2pt
  \abovedisplayshortskip=0pt plus 2pt
  \belowdisplayshortskip=4pt plus 2pt minus 2pt
}
% Times New Roman 正文 + 数学字体
\usepackage{newtxtext,newtxmath}

\usepackage{zju-homework-style-v2}
\usepackage{upgreek}
\usepackage{float}
\graphicspath{{figures/}}

\begin{document}

% --- 扉页 ---
\begin{titlepage}
    \maketitlepage
\end{titlepage}
\clearpage

% --- 作业正文 ---

% ==================== QUESTION 1 ====================
\begin{problem}
    \heiti\zihao{4}{\textcolor{QuestionBlue}{Question 1.}} (1.19) For each of the systems in Problem 1.17, determine whether the system is linear or nonlinear. Justify your answers.
\end{problem}

\vspace*{5pt}
\subsection*{\heiti\zihao{4}Solution (b): y(t) = e^{x(t)}}
To check for linearity, we test the superposition principle. Let the system operator be $T\{\cdot\}$.
Let the inputs be $x_1(t)$ and $x_2(t)$. The corresponding outputs are $y_1(t) = e^{x_1(t)}$ and $y_2(t) = e^{x_2(t)}$.

Consider a linear combination of the inputs $a x_1(t) + b x_2(t)$. The output is:
\begin{equation*}
    T\{a x_1(t) + b x_2(t)\} = e^{a x_1(t) + b x_2(t)} = e^{a x_1(t)} e^{b x_2(t)}
\end{equation*}
Now consider the linear combination of the outputs:
\begin{equation*}
    a y_1(t) + b y_2(t) = a e^{x_1(t)} + b e^{x_2(t)}
\end{equation*}
Since $e^{a x_1(t)} e^{b x_2(t)} \neq a e^{x_1(t)} + b e^{x_2(t)}$, the system does not satisfy the superposition principle.
Therefore, the system is \textbf{Nonlinear}.

\vspace*{20pt}
\subsection*{\heiti\zihao{4}Solution (e): y(t) = $\int_0^t (t-\lambda)x(\lambda)d\lambda$}
Let the system operator be $T\{\cdot\}$. Consider a linear combination of two inputs, $a x_1(t) + b x_2(t)$. The output is:
\begin{align*}
    T\{a x_1(t) + b x_2(t)\} &= \int_0^t (t-\lambda)[a x_1(\lambda) + b x_2(\lambda)]d\lambda \\
    &= a \int_0^t (t-\lambda)x_1(\lambda)d\lambda + b \int_0^t (t-\lambda)x_2(\lambda)d\lambda \\
    &= a T\{x_1(t)\} + b T\{x_2(t)\}
\end{align*}
The output from the combined input is equal to the combination of the individual outputs.
Therefore, the system is \textbf{Linear}.


% ==================== QUESTION 2 ====================


\vspace*{50pt}
\clearpage
\begin{problem}
    \heiti\zihao{4}{\textcolor{QuestionBlue}{Question 2.}} (1.20) For each of the systems in Problem 1.17, determine whether the system is time invariant or time varying. Justify your answers.
\end{problem}

\vspace*{5pt}
\subsection*{\heiti\zihao{4}Solution (b): y(t) = e^{x(t)}}
To check for time invariance, we compare the output of a delayed input with the delayed output.

\noindent\textbf{1. Output of a delayed input (Delay first, then System):}
Let the delayed input be $x_d(t) = x(t-t_0)$. The response to this input is:
\begin{equation*}
    y_1(t) = e^{x_d(t)} = e^{x(t-t_0)}
\end{equation*}

\noindent\textbf{2. Delayed output (System first, then Delay):}
Let the input be $x(t)$. The response is $y(t) = e^{x(t)}$. Delaying this output by $t_0$ gives:
\begin{equation*}
    y_2(t) = y(t-t_0) = e^{x(t-t_0)}
\end{equation*}
Since $y_1(t) = y_2(t)$, the output for a delayed input is the same as the delayed output.
Therefore, the system is \textbf{Time-Invariant}.

\vspace*{20pt}
\subsection*{\heiti\zihao{4}Solution (e): y(t) = $\int_0^t (t-\lambda)x(\lambda)d\lambda$}

\noindent\textbf{1. Output of a delayed input (Delay first, then System):}
Let the delayed input be $x_d(t) = x(t-t_0)$. The response to this input is:
\begin{equation*}
    y_1(t) = \int_0^t (t-\lambda)x_d(\lambda)d\lambda = \int_0^t (t-\lambda)x(\lambda-t_0)d\lambda
\end{equation*}

\noindent\textbf{2. Delayed output (System first, then Delay):}
Let the input be $x(t)$. The response is $y(t) = \int_0^t (t-\lambda)x(\lambda)d\lambda$. Delaying this output by $t_0$ (replacing $t$ with $t-t_0$) gives:
\begin{equation*}
    y_2(t) = y(t-t_0) = \int_0^{t-t_0} (t-t_0-\lambda)x(\lambda)d\lambda
\end{equation*}
Since the expressions for $y_1(t)$ and $y_2(t)$ are not identical, the system's behavior depends on the specific time the input is applied (due to the fixed integration limit of 0).
Therefore, the system is \textbf{Time-Varying}.


% ==================== QUESTION 3 ====================
\vspace*{50pt}
\begin{problem}
    \heiti\zihao{4}{\textcolor{QuestionBlue}{Question 3.}} (1.29) Determine whether the following discrete-time systems are causal or noncausal, have memory or are memoryless, are linear or nonlinear, are time invariant or time varying. Justify your answers. In the following parts, x[n] is an arbitrary input and y[n] is the response to x[n].
\end{problem}

\vspace*{5pt}
\subsection*{\heiti\zihao{4}Solution (a): y[n] = x[n] + 2x[n - 2]}
\textbf{Causality:} The output $y[n]$ depends on the current input $x[n]$ and the past input $x[n - 2]$. It does not depend on any future inputs. Thus, the system is \textbf{Causal}. 

\textbf{Memory:} The output depends on the past input $x[n - 2]$. Therefore, the system \textbf{has Memory}. 

\textbf{Linearity:} The system satisfies the superposition principle. It is \textbf{Linear}. 

\textbf{Time-Invariance:} The system's behavior does not depend on the specific time the input is applied. A delayed input $x[n-n_0]$ produces an output $x[n-n_0] + 2x[n-n_0-2]$. The delayed output $y[n-n_0]$ is also $x[n-n_0] + 2x[n-n_0-2]$. They are identical. Thus, the system is \textbf{Time-Invariant}.

\vspace*{20pt}
\subsection*{\heiti\zihao{4}Solution (c): y[n] = nx[n]}
\textbf{Causality:} The output $y[n]$ depends only on the current input $x[n]$. It does not depend on future inputs. Thus, the system is \textbf{Causal}.

\textbf{Memory:} The output depends only on the current input $x[n]$ and not on any past inputs. Therefore, the system is \textbf{Memoryless}.

\textbf{Linearity:} The system involves only scaling the input, so it satisfies the superposition principle. It is \textbf{Linear}.

\textbf{Time-Invariance:} Consider a delayed input $x[n-n_0]$. The output is $n x[n-n_0]$. Now consider the delayed output, which is $y[n-n_0] = (n-n_0)x[n-n_0]$. Since $n x[n-n_0] \neq (n-n_0)x[n-n_0]$, the system is \textbf{Time-Varying}.

\end{document}